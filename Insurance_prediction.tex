\documentclass{beamer}
\usepackage{graphicx}
\mode<presentation>
{
  \usetheme{Warsaw}      % or try Darmstadt, Madrid, Warsaw, ...
  %\usecolortheme{albatross} % or try albatross, beaver, crane, ...
  \usefonttheme{serif}  % or try serif, structurebold, ...
  \setbeamertemplate{navigation symbols}{}
  \setbeamertemplate{caption}[numbered]
} 


\title[]{\textbf{MEDICAL INSURANCE COST PREDICTION}}
\institute{\Large\textbf{ SHRI VISHNU ENGINEERING COLLEGE FOR WOMEN}}
\author[]{L.Harshitha 21B01A0590: CSE\\
        N.Sai Ramya Sri 21B01A05B4: CSE \\
        O.Pranathi Sudha 21B01A05C7: CSE\\
        P.Divya Tejaswini 21B01A05C9: CSE\\}
\date{17th February, 2024}

\begin{document}

\begin{frame}
  \titlepage
\end{frame}


% These three lines create an automatically generated table of contents.
\begin{frame}{Table Of Content}
 \tableofcontents
\begin{enumerate}
    \item Introduction
     \item Problem Statement
      \item Dataset Description
       \item Model Building
        \item Evaluation Metrics
         \item Deployment
          \item Accuracy Results
             \item Result    
\end{enumerate}
\end{frame}


\begin{frame}{Introduction}

\begin{itemize}
  \item The healthcare industry is undergoing various Technological adaptations. Amidst these changes, the ability to predict medical insurance costs has emerged as a critical task for insurance providers and healthcare organizations. \\
  \item The goal is to understand the factors that contribute to insurance and to build a machine learning model capable of predicting the Medical Insurance Cost.\\
  \end{itemize}
\end{frame}

\begin{frame}{\textbf{Problem Statement}}
\begin{itemize}
\item The project focuses on predicting medical insurance using a dataset.\\
\item The problem is framed as a regression task: predicting the insurance charges based on various features such as age, sex, BMI, number of children, smoking status, and region.\\
\end{itemize}



\end{frame}

\begin{frame}{\textbf {Dataset Description}}

\begin{itemize}

\item The Dataset includes information such as patient demographics, region etc.\\
\item  Key features include age, gender, BMI, region, smoker or not, number of children.\\
\item  The target variable is "Insurance Money," and the Machine Learning model can be trained on historical data to predict this outcome for new individuals.\\
 \end{itemize}
\end{frame}

\begin{frame}{\textbf{Model Building}}

  Three classification models have been implemented in the project:
\begin{itemize}
    \item Linear Regression\\
   \item Decision Tree Regression\\
   \item Random Forest Regression\\
\end{itemize}

\end{frame}
\begin{frame}{Evaluation Metrics}
  The performance of the models is assessed using common classification metrics:
\begin{itemize}
    \item Accuracy\\
   \item Root Mean Square Error\\
   \item Mean Absolute Error\\
\end{itemize}
\end{frame}


\begin{frame}{Deployment}
\begin{itemize}
    \item Framework : gradio \\
    \item Programming Language: Python \\
     \item Version Control : Git hub \\
\end{itemize}
\end{frame}


\begin{frame}{Accuracy Results}
    \begin{table}
\centering
\caption{\large \textbf{Models with Accuracy}}
\begin{tabular}{|c|c|c|}
\hline
\textbf{Model} & \textbf{Accuracy}  \\
\hline
Linear Regression & 0.727 \\
\hline
RandmForest Regression & 0.936 \\
\hline
Decision Tree Regressor & 0.916 \\

\hline
\end{tabular}
\end{table}
\begin{itemize}
    \item Selected Model : Decision Tree Regressor\\
\end{itemize} 
\end{frame}


\begin{frame}{Random Forest Regression Accuracy}
    \includegraphics[width=1\textwidth]{rf.png }
\end{frame}


\begin{frame}{Decision Tree Regressor Accuracy}
    \includegraphics[width=1\textwidth]{dt.png}
\end{frame}

\begin{frame}{Result}
    \includegraphics[width=1\textwidth]{output.png}
\end{frame}

\begin{frame}\begin{center}
\Huge {Thank You}
\end{center}
   
\end{frame}
 \end{document}
\end{document}